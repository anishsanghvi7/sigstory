\nonstopmode{}
\documentclass[a4paper]{book}
\usepackage[times,inconsolata,hyper]{Rd}
\usepackage{makeidx}
\makeatletter\@ifl@t@r\fmtversion{2018/04/01}{}{\usepackage[utf8]{inputenc}}\makeatother
% \usepackage{graphicx} % @USE GRAPHICX@
\makeindex{}
\begin{document}
\chapter*{}
\begin{center}
{\textbf{\huge Package `sigstory'}}
\par\bigskip{\large \today}
\end{center}
\ifthenelse{\boolean{Rd@use@hyper}}{\hypersetup{pdftitle = {sigstory: What the Package Does (Short Line)}}}{}
\begin{description}
\raggedright{}
\item[Type]\AsIs{Package}
\item[Title]\AsIs{What the Package Does (Short Line)}
\item[Version]\AsIs{1.0}
\item[Date]\AsIs{2024-07-29}
\item[Author]\AsIs{Who wrote it}
\item[Maintainer]\AsIs{Who to complain to }\email{yourfault@somewhere.net}\AsIs{}
\item[Description]\AsIs{More about what it does (maybe more than one line).}
\item[License]\AsIs{What license is it under?}
\item[RoxygenNote]\AsIs{7.3.1}
\item[NeedsCompilation]\AsIs{no}
\end{description}
\Rdcontents{Contents}
\HeaderA{sigstory-package}{What the Package Does (Short Line)}{sigstory.Rdash.package}
\aliasA{sigstory}{sigstory-package}{sigstory}
\keyword{package}{sigstory-package}
%
\begin{Description}
More about what it does (maybe more than one line).
\end{Description}
%
\begin{Author}
Who wrote it

Maintainer: Who to complain to <yourfault@somewhere.net>
\end{Author}
\HeaderA{generate\_single\_report}{A Capitalized Title (ideally limited to 65 characters)}{generate.Rul.single.Rul.report}
%
\begin{Usage}
\begin{verbatim}
generate_single_report(catalogue, bootstraps, bootstraps_experimental, similarity, tally, dataset, parquet_path = NULL, sample_information = NULL)
\end{verbatim}
\end{Usage}
%
\begin{Arguments}
\begin{ldescription}
\item[\code{catalogue}] 

\item[\code{bootstraps}] 

\item[\code{bootstraps\_experimental}] 

\item[\code{similarity}] 

\item[\code{tally}] 

\item[\code{dataset}] 

\item[\code{parquet\_path}] 

\item[\code{sample\_information}] 

\end{ldescription}
\end{Arguments}
%
\begin{Examples}
\begin{ExampleCode}
##---- Should be DIRECTLY executable !! ----
##-- ==>  Define data, use random,
##--	or standard data sets, see data().

## The function is currently defined as
function (catalogue, bootstraps, bootstraps_experimental, similarity, 
    tally, dataset, parquet_path = NULL, sample_information = NULL) 
{
    expo_file <- catalogue
    bootstrap_file <- bootstraps
    bootstraps_experimental_file <- bootstraps_experimental
    tally_file <- tally
    similarity_file <- similarity
    parquet_folder <- parquet_path
    sample_info_file <- sample_information
    split_catalogue <- stringr::str_split(catalogue, "\.")[[1]]
    sample_of_interest_cat <- split_catalogue[2]
    split_bootstraps <- stringr::str_split(bootstraps, "\.")[[1]]
    sample_of_interest_boot <- split_catalogue[2]
    split_tally <- stringr::str_split(tally, "\.")[[1]]
    sample_of_interest_tally <- split_catalogue[2]
    if (sample_of_interest_cat != sample_of_interest_boot || 
        sample_of_interest_cat != sample_of_interest_tally || 
        sample_of_interest_boot != sample_of_interest_tally) {
        stop("Input files do not come from the same sample")
    }
    sample_of_interest <- sample_of_interest_cat
    pattern_type <- c("SBS96", "DBS78", "ID83")
    if (grepl(pattern_type[1], expo_file)) {
        sig_type <- "SBS96"
    }
    else if (grepl(pattern_type[2], expo_file)) {
        sig_type <- "DBS78"
    }
    else if (grepl(pattern_type[3], expo_file)) {
        sig_type <- "ID83"
    }
    else {
        stop("Input files do not have a mutation type in the filename")
    }
    try(fs::dir_create(paste0("/results/", sample_of_interest)))
    rmarkdown::render(input = "/vignettes/SignatureAnalysis_Full.Rmd", 
        output_format = "html_document", output_file = paste0("/results/", 
            sample_of_interest, "/MutationalSignatureAnalysis_", 
            sample_of_interest, "_", sig_type, ".html"), params = list(sample_of_interest = sample_of_interest, 
            sig_type = sig_type, sig_description = dataset, catalogue = expo_file, 
            tally = tally_file, bootstraps = bootstrap_file, 
            experimental_bootstraps = bootstraps_experimental_file, 
            sample_similarity = similarity_file, parquet_file_path = parquet_folder, 
            sample_information = sample_info_file))
  }
\end{ExampleCode}
\end{Examples}
\HeaderA{generate\_summary\_layer}{A Capitalized Title (ideally limited to 65 characters)}{generate.Rul.summary.Rul.layer}
%
\begin{Usage}
\begin{verbatim}
generate_summary_layer(catalogue, bootstraps, tally, dataset, catalogue2, bootstraps2, tally2, dataset2, catalogue3, bootstraps3, tally3, dataset3)
\end{verbatim}
\end{Usage}
%
\begin{Arguments}
\begin{ldescription}
\item[\code{catalogue}] 

\item[\code{bootstraps}] 

\item[\code{tally}] 

\item[\code{dataset}] 

\item[\code{catalogue2}] 

\item[\code{bootstraps2}] 

\item[\code{tally2}] 

\item[\code{dataset2}] 

\item[\code{catalogue3}] 

\item[\code{bootstraps3}] 

\item[\code{tally3}] 

\item[\code{dataset3}] 

\end{ldescription}
\end{Arguments}
%
\begin{Examples}
\begin{ExampleCode}
##---- Should be DIRECTLY executable !! ----
##-- ==>  Define data, use random,
##--	or standard data sets, see data().

## The function is currently defined as
function (catalogue, bootstraps, tally, dataset, catalogue2, 
    bootstraps2, tally2, dataset2, catalogue3, bootstraps3, tally3, 
    dataset3) 
{
    split_catalogue <- stringr::str_split(catalogue, "\.")[[1]]
    sample_of_interest_cat <- split_catalogue[2]
    split_bootstraps <- stringr::str_split(bootstraps, "\.")[[1]]
    sample_of_interest_boot <- split_catalogue[2]
    split_tally <- stringr::str_split(tally, "\.")[[1]]
    sample_of_interest_tally <- split_catalogue[2]
    if (sample_of_interest_cat != sample_of_interest_boot || 
        sample_of_interest_cat != sample_of_interest_tally || 
        sample_of_interest_boot != sample_of_interest_tally) {
        stop("Input files do not come from the same sample")
    }
    sample_of_interest <- sample_of_interest_cat
    try(fs::dir_create(paste0("/results/", sample_of_interest)))
    rmarkdown::render(input = "/vignettes/SignatureAnalysis_Summary.Rmd", 
        output_format = "html_document", output_file = paste0("/results/", 
            sample_of_interest, "/MutationalSignatureAnalysis_", 
            sample_of_interest, "_Summary.html"), params = list(sample_of_interest = sample_of_interest, 
            cos_threshold = 0.9, sig_description_sbs96 = dataset, 
            sig_description_dbs78 = dataset2, sig_description_id83 = dataset3, 
            catalogue_sbs96 = catalogue, tally_sbs96 = tally, 
            bootstraps_sbs96 = bootstraps, catalogue_dbs78 = catalogue2, 
            tally_dbs78 = tally2, bootstraps_dbs78 = bootstraps2, 
            catalogue_id83 = catalogue3, tally_id83 = tally3, 
            bootstraps_id83 = bootstraps3))
  }
\end{ExampleCode}
\end{Examples}
\HeaderA{sigstory}{A Capitalized Title (ideally limited to 65 characters)}{sigstory}
%
\begin{Usage}
\begin{verbatim}
sigstory(catalogue, bootstraps, bootstraps_experimental, similarity, tally, dataset, parquet_path = NULL, catalogue2 = NULL, bootstraps2 = NULL, bootstraps_experimental2 = NULL, similarity2 = NULL, tally2 = NULL, dataset2 = NULL, parquet_path2 = NULL, catalogue3 = NULL, bootstraps3 = NULL, bootstraps_experimental3 = NULL, similarity3 = NULL, tally3 = NULL, dataset3 = NULL, parquet_path3 = NULL, sample_information = NULL)
\end{verbatim}
\end{Usage}
%
\begin{Arguments}
\begin{ldescription}
\item[\code{catalogue}] 

\item[\code{bootstraps}] 

\item[\code{bootstraps\_experimental}] 

\item[\code{similarity}] 

\item[\code{tally}] 

\item[\code{dataset}] 

\item[\code{parquet\_path}] 

\item[\code{catalogue2}] 

\item[\code{bootstraps2}] 

\item[\code{bootstraps\_experimental2}] 

\item[\code{similarity2}] 

\item[\code{tally2}] 

\item[\code{dataset2}] 

\item[\code{parquet\_path2}] 

\item[\code{catalogue3}] 

\item[\code{bootstraps3}] 

\item[\code{bootstraps\_experimental3}] 

\item[\code{similarity3}] 

\item[\code{tally3}] 

\item[\code{dataset3}] 

\item[\code{parquet\_path3}] 

\item[\code{sample\_information}] 

\end{ldescription}
\end{Arguments}
%
\begin{Examples}
\begin{ExampleCode}
##---- Should be DIRECTLY executable !! ----
##-- ==>  Define data, use random,
##--	or standard data sets, see data().

## The function is currently defined as
function (catalogue, bootstraps, bootstraps_experimental, similarity, 
    tally, dataset, parquet_path = NULL, catalogue2 = NULL, bootstraps2 = NULL, 
    bootstraps_experimental2 = NULL, similarity2 = NULL, tally2 = NULL, 
    dataset2 = NULL, parquet_path2 = NULL, catalogue3 = NULL, 
    bootstraps3 = NULL, bootstraps_experimental3 = NULL, similarity3 = NULL, 
    tally3 = NULL, dataset3 = NULL, parquet_path3 = NULL, sample_information = NULL) 
{
    if (!is.null(catalogue2) && (catalogue == catalogue2 || (!is.null(catalogue3) && 
        catalogue2 == catalogue3) || (!is.null(catalogue3) && 
        catalogue == catalogue3))) {
        stop("Input catalogue files are the same")
    }
    if (!is.null(bootstraps2) && (bootstraps == bootstraps2 || 
        (!is.null(bootstraps3) && bootstraps2 == bootstraps3) || 
        (!is.null(bootstraps3) && bootstraps == bootstraps3))) {
        stop("Input bootstraps files are the same")
    }
    if (!is.null(bootstraps_experimental2) && (bootstraps_experimental == 
        bootstraps_experimental2 || (!is.null(bootstraps_experimental3) && 
        bootstraps_experimental == bootstraps_experimental3) || 
        (!is.null(bootstraps_experimental3) && bootstraps_experimental == 
            bootstraps_experimental3))) {
        stop("Input experimental bootstrap files are the same")
    }
    if (!is.null(similarity2) && (similarity == similarity2 || 
        (!is.null(similarity3) && similarity2 == similarity3) || 
        (!is.null(similarity3) && similarity == similarity3))) {
        stop("Input tally files are the same")
    }
    if (!is.null(dataset2) && (dataset == dataset2 || (!is.null(dataset3) && 
        dataset2 == dataset3) || (!is.null(dataset3) && dataset == 
        dataset3))) {
        stop("Input COSMIC datasets files are the same")
    }
    if (is.null(catalogue2) || is.null(bootstraps2) || is.null(tally2) || 
        is.null(dataset2) || is.null(catalogue3) || is.null(bootstraps3) || 
        is.null(tally3) || is.null(dataset3)) {
        expo_file <- catalogue
        bootstrap_file <- bootstraps
        bootstraps_experimental_file <- bootstraps_experimental
        tally_file <- tally
        sample_file <- sample_information
        generate_single_report(expo_file, bootstrap_file, bootstraps_experimental_file, 
            similarity, tally_file, dataset, parquet_path, sample_file)
    }
    else if (!is.null(catalogue2) && !is.null(bootstraps2) && 
        !is.null(tally2) && !is.null(dataset2) && !is.null(catalogue3) && 
        !is.null(bootstraps3) && !is.null(tally3) && !is.null(dataset3)) {
        sample_file <- sample_information
        expo_file <- catalogue
        bootstrap_file <- bootstraps
        bootstraps_experimental_file <- bootstraps_experimental
        tally_file <- tally
        generate_single_report(expo_file, bootstrap_file, bootstraps_experimental_file, 
            similarity, tally_file, dataset, parquet_path, sample_file)
        expo_file2 <- catalogue2
        bootstrap_file2 <- bootstraps2
        bootstraps_experimental_file2 <- bootstraps_experimental2
        tally_file2 <- tally2
        generate_single_report(expo_file2, bootstrap_file2, bootstraps_experimental_file2, 
            similarity2, tally_file2, dataset2, parquet_path2, 
            sample_file)
        expo_file3 <- catalogue3
        bootstrap_file3 <- bootstraps3
        bootstraps_experimental_file3 <- bootstraps_experimental3
        tally_file3 <- tally3
        generate_single_report(expo_file3, bootstrap_file3, bootstraps_experimental_file3, 
            similarity3, tally_file3, dataset3, parquet_path3, 
            sample_file)
        generate_summary_layer(catalogue, bootstraps, tally, 
            dataset, catalogue2, bootstraps2, tally2, dataset2, 
            catalogue3, bootstraps3, tally3, dataset3)
    }
    else {
        stop("No files were created, ensure no NULL values are being passed in")
    }
  }
\end{ExampleCode}
\end{Examples}
\printindex{}
\end{document}
